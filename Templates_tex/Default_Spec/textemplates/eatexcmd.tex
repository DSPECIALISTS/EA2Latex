
% Diese Datei definiert die vom EA2Latex verwendeten Kommandos.

% Define url command "filepath" to be used with file paths
\DeclareUrlCommand\filepath{\urlstyle{sf}}
\urlstyle{sf}

% Macros to access URL part of references
\def\refpartfive#1#2#3#4#5{%
#5%
}

\def\eatexdocname#1#2#3#4#5{%
	\expandafter\ifx\csname eatexdoc#5\endcsname\relax%
%		?docname?%
	\else%
		\expandafter\csname eatexdoc#5\endcsname%
	\fi%
}


\newcommand{\urlref}[1]
{%
\expandafter\expandafter\expandafter\refpartfive\csname r@#1\endcsname{}%
}
\def\docref#1{%
	\expandafter\ifx\csname r@#1\endcsname\relax%
	\else%
	\textcolor{MidnightBlue}{\expandafter\expandafter\expandafter\eatexdocname\csname r@#1\endcsname}%
	\fi%
}
\makeatother

\makeatletter
\renewcommand\paragraph{\@startsection{paragraph}{4}{\z@}%
  {-3.25ex\@plus -1ex \@minus -.2ex}%
  {1.5ex \@plus .2ex}%
  {\normalfont\normalsize\bfseries}}
\makeatother

\makeatletter
\renewcommand\subparagraph{\@startsection{subparagraph}{5}{\z@}%
  {-3.25ex\@plus -1ex \@minus -.2ex}%
  {1.5ex \@plus .2ex}%
  {\normalfont\normalsize\bfseries}}
\makeatother

\makeatletter
\renewcommand\subsubparagraph{\@startsection{subsubparagraph}{6}{\z@}%
  {-3.25ex\@plus -1ex \@minus -.2ex}%
  {1.5ex \@plus .2ex}%
  {\normalfont\normalsize\bfseries}}
\makeatother

%---------------------------------------------------------%
%%%%%%%% DSPE commands %%%%%%%%%
%---------------------------------------------------------%
% Ueberschriften
\newcommand{\DSPEflatsect}[1]%
{%
	\DSPEelementheader{#1}\newline%
}%
\newcommand{\DSPEsectlvlzero}[2][]{\ifthenelse{\equal{#1}{}}{\section{#2}}{\section[#1]{#2}}}
\newcommand{\DSPEsectlvlone}[2][]{\ifthenelse{\equal{#1}{}}{\subsection{#2}}{\subsection[#1]{#2}}}
\newcommand{\DSPEsectlvltwo}[2][]{\ifthenelse{\equal{#1}{}}{\subsubsection{#2}}{\subsubsection[#1]{#2}}}
\newcommand{\DSPEsectlvlthree}[2][]%
{%
	\ifthenelse{\equal{#1}{}}{\paragraph{#2}}{\paragraph[#1]{#2}}%
	
}
\newcommand{\DSPEsectlvlfour}[2][]%
{%
	\ifthenelse{\equal{#1}{}}{\subparagraph{#2}}{\subparagraph[#1]{#2}}%
}
\newcommand{\DSPEsectlvlfive}[2][]
{%
	\textbf{#2}\\%
}
\newcommand{\DSPEsectlvlsix}[2][]{\DSPEsectlvlfive[#1]{#2}}
\newcommand{\DSPEsectlvlseven}[2][]{\DSPEsectlvlfive[#1]{#2}}
\newcommand{\DSPEsectlvleight}[2][]{\DSPEsectlvlfive[#1]{#2}}
\newcommand{\DSPEsectlvlnine}[2][]{\DSPEsectlvlfive[#1]{#2}}
\newcommand{\DSPEsectlvlten}[2][]{\DSPEsectlvlfive[#1]{#2}}
% --------------- Package Elements --------------%

%package
% #1 Package Name
% #2 Level
% #3 Package Notes
% #4 Guid
% #5 section style
% #6 Stereotype
% #7 Status
\newcommand{\DSPEpackage}[7]%
{%
	\ifthenelse{\equal{#5}{tree}}
	{
		\ifcase #2
		\DSPEsectlvlzero{#1}
		\or
		\DSPEsectlvlone{#1}
		\or
	        \DSPEsectlvltwo{#1}
	        \or
	        \DSPEsectlvlthree{#1}
	        \or
	        \DSPEsectlvlfour{#1}
	        \or
	        \DSPEsectlvlfive{#1}
	        \else
	        \DSPEsectlvlfive{#1}
		\fi
	}
	{
		\DSPEflatsect{#1}
	}%
	\label{#4}%
	#3%
}


\newcommand{\DSPEelementheader}[1]%
{%
	\subsubparagraph*{\textcolor{MidnightBlue}{#1}}%
}

% Erhaelt die Argumente:
% #1 Element-Name
% #2 Element-Type
% #3 Element-Notes
% #4 Guid
% #5 section_style
% #6 level
% #7 Stereotype
% #8 Status
\newcommand{\DSPEelement}[8]%
{%
	\ifthenelse{\not\equal{#2}{Note}}%
	{%
		\ifthenelse{\equal{#5}{flat}}
		{
			\DSPEelementheader{#1 (#2)}
		}
		{
			\ifcase #6
	                \DSPEsectlvlzero[#1]{#1 (#2)}
	                \or
	                \DSPEsectlvlone[#1]{#1 (#2)}
	                \or
	                \DSPEsectlvltwo[#1]{#1 (#2)}
	                \or
	                \DSPEsectlvlthree[#1]{#1 (#2)}
	                \or
	                \DSPEsectlvlfour[#1]{#1 (#2)}
	                \or
	                \DSPEsectlvlfive[#1]{#1 (#2)}
	                \else
	                \DSPEsectlvlfive[#1]{#1 (#2)}
	               	\fi
		}
	\label{#4}%
	#3%
	}{}%
}
% Erhaelt die Argumente:
% #1 Element-Name
% #2 Element-Notes
% #3 Dateipfad
% #4 Guid
% #5 section_style
% #6 level
% #7 Stereotype
% #8 Status
\newcommand{\DSPEartifact}[8]%
{%
	\ifthenelse{\equal{#5}{flat}}
        {
                \DSPEelementheader{#1}
        }
        {
                \ifcase #6
                \DSPEsectlvlzero{#1}
                \or
                \DSPEsectlvlone{#1}
                \or
                \DSPEsectlvltwo{#1}
                \or
                \DSPEsectlvlthree{#1}
                \or
                \DSPEsectlvlfour{#1}
                \or
                \DSPEsectlvlfive{#1}
                \else
                \DSPEsectlvlfive{#1}
                \fi
        }%
	\label{#4}%
	\ifthenelse{\not\equal{#3}{}}%
	{%
		\footnotesize \langStrFile: \expandafter\expandafter\expandafter\filepath\expandafter{#3} \normalsize\\%
	}{}%
	#2

}

% Erhaelt die Argumente:
% #1 Element-Name
% #2 Element-Notes
% #3 Guid
% #4 section_style
% #5 level
% #6 Stereotype
% #7 Status
\newcommand{\DSPEcollab}[7]{\DSPEelement{#1}{Collaboration}{#2}{#3}{#4}{#5}{#6}{#7}}

% Erhaelt die Argumente:
% #1 Element-Name
% #2 classifier_elem_name
% #3 Element-Notes
% #4 Guid
% #5 section_style
% #6 level
% #7 Stereotype
% #8 Status
\newcommand{\DSPEcomponent}[8]
{%
	\DSPEelement{\DSPEinstanceheader{#1}{#2}}{Component}{#3}{#4}{#5}{#6}{#7}{#8}
}
% Header für das Component-Element
% Wenn kein Typ angegeben, dann keine Doppelpunkt
\newcommand{\DSPEinstanceheader}[2]%
{%
#1\if\relax\detokenize{#2}\relax%
\else%
:#2%
\fi%
}

% Erhaelt die Argumente:
% #1 Element-Name
% #2 classifier_name
% #3 Element-Notes
% #4 Guid
% #5 section_style
% #6 level
% #7 Stereotype
% #8 Status
\newcommand{\DSPEobject}[8]{\DSPEelement{\DSPEinstanceheader{#1}{#2}}{Object}{#3}{#4}{#5}{#6}{#7}{#8}}

% Erhaelt die Argumente:
% #1 Klassen-Name
% #2 Klassen-Notes
% #3 GUID
% #4 section_style
% #5 level
% #6 Stereotype
% #7 Status
\newcommand{\DSPEclasselem}[7]
{%
	\DSPEelement{#1}{\if\relax#6\relax{}Class\else#6\fi}{#2}{#3}{#4}{#5}{#6}{#7}
}

% Erhaelt die Argumente:
% #1 Klassen-Name
% #2 Klassen-Notes
% #3 GUID
% #4 section_style
% #5 level
% #6 Stereotype
% #7 Status
\newcommand{\DSPEinterface}[7]{\DSPEelement{#1}{Interface}{#2}{#3}{#4}{#5}{#6}{#7}}

\newcommand{\DSPEimgcomponent}[2]%
{%
	\subparagraph*{\textcolor{MidnightBlue}{#1}}%

	#2%
}

% ---------- Element "State" Umgebung ---------%

% Erhaelt die Argumente:
% #1 Name
% #2 Notes
% #3 GUID
% #4 section_style
% #5 level
% #6 Stereotype
% #7 Status
\newcommand{\DSPEstate}[7]{\DSPEelement{#1}{State}{#2}{#3}{#4}{#5}{#6}{#7}}

% erhält in der Reihenfolge den Scenario-Namen, die Notes, den Type und den ObjectType
\newcommand{\DSPEscenario}[4]
{%
	\hspace*{\fill}\newline
	\textbf{#1}\\
%	#2 #3 #4\\
}
% Erhält in der Reihenfolge den Step-Namen, das Level, die Position und den State 
\newcommand{\DSPEscenariostep}[4]
{%
	#2. #1\\
%	#3 #4\\
}

% Erhält  als Argumente:
% #1 Extension-Name
% #2 Level
% #3 Typ
\newcommand{\DSPEscenarioext}[3]
{%
	#2. #3: \textbf{#1}\\
}

% -------- Element "Requirement" Umgebung --------%
% Erhaelt als Argumente:
% #1 Requirement Namen
% #2 Requirement Notes
% #3 Stereotype (z.B. functional)
% #4 Status (z.B. proposed)
% #5 Difficulty
% #6 Priority
% #7 Requirement Nummer (z.B. REQ-23)
% #8 Guid
\newcommand{\DSPErequirement}[8]
{%
	% Alternative 1
	\subsubsection*{#1 \textnormal{(#3, #4)}}\label{#8}
	#2			
	%	Alternative 2
%	\begin{table}[tbp]
%		\centering
%		\begin{tabular}{|v{0.3\textwidth}|v{0.65\textwidth}|}
%			\hline
%			\multicolumn{2}{|l|}{\textbf{Anforderung}} \tabularnewline \hline\hline
%			Kurzbeschreibung & #1 \tabularnewline \hline
%			Typ                     & #3 \tabularnewline \hline
%			Status                 & #4 \tabularnewline \hline
%			Schwierigkeit       & #5 \tabularnewline \hline
%			Priorität               & #6 \tabularnewline \hline
%		\end{tabular}
%		\caption{Anforderung #7}
%		\label{#8}
%	\end{table}
}


% ------------- Figures ------------------%
% Erhaelt als Argumente:
% #1 Guid
% #2 Diagramm-Name
% #3 scaling
% #4 Diagram Notes
% #5 Stereotype
% Wird als drittes Argument "scaled" uebergeben, dann wurde das Diagramm bereits
% im EA auf Seitengroesse skaliert und es wird die maximale Groesse bzw. Breite
% benutzt. Wenn "fit", dann passt das Diagramm auf die Seite und es wird nur ein 
% konstanter Skalierungsfaktor benutzt, damit die Seitenraender beruecksichtigt 
% werden.
\newcommand{\DSPEfigure}[5]%
{%
	\ifthenelse{\equal{#3}{scaled}}%
	{%
		\begin{figure}[H]%
			\centering%
			\includegraphics[width=\textwidth,height=\textheight, keepaspectratio]{#1}%
			\ifthenelse{\equal{#4}{ }}{}%
			{%
				\newline%
				\DSPEnote{#4}%
			}%
			\caption{#2}%
			\label{#1}%
		\end{figure}%
	}%
	{%
        \begin{figure}[H]%
            \centering%
            \includegraphics[scale=0.8]{#1}%
			\ifthenelse{\equal{#4}{ }}{}%
			{%	
				\newline%
				\DSPEnote{#4}%
			}%
            \caption{#2}%
            \label{#1}%
        \end{figure}%
	}%
}
% Befehl um externe PDF PNG JPG EPS Dateien anzuzeigen
% Erhält als Argument
% #1 Dateipfad
% #2 Notes
% #3 Name 
% #4 Guid
\newcommand{\DSPEextfigure}[4]%
{%
	\DSPEfigurebegin%
	\includegraphics[width=\textwidth,height=\textheight, keepaspectratio]{#1}%
	\newline%
	\DSPEfigureend{#2}{#3}{#4}%
}

%Befehle um Diagramm aus linked Documents zu referenzieren bzw anzuzeigen
\newcommand{\DSPEfigurebegin}%
{%
	\begin{figure}[H]%
	\centering%
}
% Erhaelt als Argumente:
% #1 Notes
% #2 Diagramm-Name
% #3 Guid

\newcommand{\DSPEfigureend}[3]%
{%	
	\ifthenelse{\equal{#1}{ }}{}%
	{%
		\DSPEnote{#1}%
	}
	\caption{#2}%
	\label{#3}%
	\end{figure}%
}
% Notes für Tabellen & Abbildungen
% Erhält als Argument
% #1 Notes
\newcommand{\DSPEnote}[1]%
{%
	\newlength{\mywidth}% Neuer Längenbefehl
	\settowidth{\mywidth}{\textit{#1}}% \mywidhth = Länge des kursiven Text
	\ifthenelse{\lengthtest{\mywidth > 0.8\textwidth}}% 
	{%
	% Wenn Länge des Text größer als 0.8\textwidth dann ist dies die Boxbreite
		\medskip%
		\parbox[t]{0.8\textwidth}{\textit{#1}}%
	}%
	{%
	% Wenn Länge des Text kleiner als 0.8\textwidth dass ist \mywidth die Boxbreite
		\medskip%
		\parbox{\mywidth}{\textit{#1}}%
	}%
}
% ------------ cross references ---------%
% Erhaelt das folgende Argument
% #1 Guid
\newcommand{\DSPEimgref}[1]%
{%
	\autoref{#1}~\frq\nameref{#1}\flq\docref{#1}%
}

\newcommand{\DSPEobjref}[1]
{%
}

% Erhaelt die folgenden Argumente
% #1 Typ des Elements auf das verwiesen wird(Package, Element, ...)
% #2 Name des Elements auf das verwiesen wird
% #3 Guid des Elements
% #4 Text der in der Referenz steht
% #5 Name des Elternelements
% #6 Stereotype
\newcommand{\DSPEelemref}[6]%
{%
	\textbf{\hyperref[#3]{#4}}\footnote{%
        \ifthenelse{\equal{#1}{Package}}%
        {%
                \langStrSeeSection{} \ref{#3}~\nameref{#3}, \langStrPage{}~\pageref{#3}%
        }{% Package else start
        \ifthenelse{\equal{#1}{Attribute}}%
        {%
                \langStrSeeTable{}~\ref{#3} #5 \frqq{} #2 (\langStrAttribute), \langStrPage{}~\pageref{#3}%
        }{% Attribute else start
        \ifthenelse{\equal{#1}{Method}}%
        {%
                \langStrSeeTable{}~\ref{#3} #5 \frqq{} #2 (\langStrMethod), \langStrPage{}~\pageref{#3}%
        }{% Method else start
		\ifthenelse{\equal{#1}{Artifact}}%
		{%
			\ifthenelse{\equal{#6}{table}}%
			{%
				\langStrSeeTable{} \ref{#3}~\nameref{#3}, \langStrPage{}~\pageref{#3}%
			}{% Table else start
                \langStrSeeSection{}~\ref{#3} \frqq{} #2, \langStrPage{}~\pageref{#3}%
			}% Table else end
		}{% Artifact else start
		    \langStrSeeSection{}~\ref{#3} \frqq{} #2, \langStrPage{}~\pageref{#3}%
		}% Artifact else end
		}% Method else end
        }% Attribute else end
        }% Package else end
	~\docref{#3}%
	}%Footnote end
}

% Erhaelt die folgenden Argumente
% #1 Typ des Elements auf das verwiesen wird(Package, Element, ...)
% #2 Name des Elements auf das verwiesen wird
% #3 Guid des Elements
% #4 Text der in der Referenz steht
% #5 Name des Elternelements
\newcommand{\DSPEextdocref} [5]%
{%
	\textbf{{#4}}\footnote{%
		\ifthenelse{\equal{#1}{ExternalDocument}}%
        {%
            \langStrSee{} #2%
        }{% ExternalDocument else start
		    \langStrSeeSection{}~\ref{#3} \frqq{} #2, \langStrPage{}~\pageref{#3}%
        }% ExternalDocument else end
	}%Footnote end
}


\newcommand{\DSPEwebsiteref}[2]%
{%
	#2\footnote{\langStrSee{} \url{#1}}%
}
%---------------- text colours ------------%
%Erhält als Argumnete:
% #1 Rot Wert 0..255
% #2 Grün-Wert 0..255
% #3 Blau-Wert 0..255
% #4 Text
\newcommand{\DSPEcolor}[4]%
{%
	\textcolor[RGB]{#1,#2,#3}{#4}%
}


%----------- TABELLEN ------------------%
\newcolumntype{v}[1]%
{%
	>{\raggedright\hspace{0pt}}p{#1}%
}

% Begin der Table-Umgebung
% Erhaelt als Argumente
% #1 Tabellentyp
% #2 Caption
% #3 Guid
% #4 Stereotype
% #5 Spaltentitel #1
% #6 Spaltentitel #2
% #7 Spaltentitel #3
\newcommand{\DSPEtablebegin}[7]%
{%
	\ifthenelse{\equal{#1}{attribute}}%
	{%	
		\DSPEtablebeginattribute{#1}{#2}{#3}{#4}{#5}{#6}{#7}%	
	}{}%
    \ifthenelse{\equal{#1}{attributeAlias}}%
    {%
		\DSPEtablebeginattributeAlias{#1}{#2}{#3}{#4}{#5}{#6}{#7}%	
    }{}%
    \ifthenelse{\equal{#1}{attributeAliasConstr}}%
    {%
		\DSPEtablebeginattributeAliasConstr{#1}{#2}{#3}{#4}{#5}{#6}{#7}%	
    }{}%
    \ifthenelse{\equal{#1}{attributeConstraint}}%
    {%
		\DSPEtablebeginattributeConstraint{#1}{#2}{#3}{#4}{#5}{#6}{#7}%	
    }{}%
	\ifthenelse{\equal{#1}{method}}%
	{%	
		\DSPEtablebeginmethod{#1}{#2}{#3}{#4}{#5}{#6}{#7}%
	}{}%
	\ifthenelse{\equal{#1}{reqtrace}}%
	{%
		\DSPEtablebeginreqtrace{#1}{#2}{#3}{#4}{#5}{#6}{#7}%
	}{}%
    \ifthenelse{\equal{#1}{runstate}}%
    {%
		\DSPEtablebeginrunstate{#1}{#2}{#3}{#4}{#5}{#6}{#7}%
    }{}%
	\ifthenelse{\equal{#1}{search2}}%
    {%
		\DSPEtablebeginsearchtwo{#1}{#2}{#3}{#4}{#5}{#6}{#7}%
    }{}%
	\ifthenelse{\equal{#1}{search2landscape}}%
    {%
		\DSPEtablebeginsearchtwolandscape{#1}{#2}{#3}{#4}{#5}{#6}{#7}%
    }{}%
	\ifthenelse{\equal{#1}{search3}}%
    {%
		\DSPEtablebeginsearchthree{#1}{#2}{#3}{#4}{#5}{#6}{#7}%				
    }{}%
	\ifthenelse{\equal{#1}{search3landscape}}%
    {%
		\DSPEtablebeginsearchthreelandscape{#1}{#2}{#3}{#4}{#5}{#6}{#7}%	
    }{}%	
}

\newcommand{\DSPEtablebeginattribute}[7]%
{%
	\ifthenelse{\equal{#4}{command interface}}%
	{%
		\begin{small}%
		\begin{longtable}{|v{0.3\textwidth}|v{0.65\textwidth}|}%
		\caption{#2 \langStrParameter}%
		\label{#3}%
		\hline%
		\textbf{\langStrParameter} & \textbf{\langStrDescription}\tabularnewline \hline\hline%
		\endfirsthead%
		\hline%
		\textbf{\langStrParameter} & \textbf{\langStrDescription}\tabularnewline \hline\hline \endhead%
	}{%
		\begin{small}%
		\begin{longtable}{|v{0.3\textwidth}|v{0.65\textwidth}|}%
		\caption{#2 \langStrAttributes}%
		\label{#3}%
		\hline%
		\textbf{\langStrAttribute} & \textbf{\langStrNotes}\tabularnewline \hline\hline%
		\endfirsthead%
		\hline%
		\textbf{\langStrAttribute} & \textbf{\langStrNotes}\tabularnewline \hline\hline \endhead%
	}%
}

\newcommand{\DSPEtablebeginattributeAlias}[7]%
{%
	\begin{small}%
	\begin{longtable}{|v{0.3\textwidth}|v{0.3\textwidth}|v{0.35\textwidth}|}%
	\caption{#2 \langStrAttributes}%
	\label{#3}%
	\hline%
	\textbf{\langStrAttribute} & \textbf{Alias}  & \textbf{\langStrNotes}\tabularnewline \hline\hline%
	\endfirsthead%
    \hline%
	\textbf{\langStrAttribute} & \textbf{Alias}  & \textbf{\langStrNotes}\tabularnewline \hline\hline \endhead%
}

\newcommand{\DSPEtablebeginattributeAliasConstr}[7]%
{%
	\begin{small}%
    \begin{longtable}{|v{0.25\textwidth}|v{0.2\textwidth}|v{0.25\textwidth}|v{0.25\textwidth}|}%
    \caption{#2 \langStrAttributes}%
    \label{#3}%
    \hline%
    \textbf{\langStrAttribute} & \textbf{Alias}  & \textbf{Constraints} & \textbf{\langStrNotes}\tabularnewline \hline\hline%
    \endfirsthead%
	\hline%
	\textbf{\langStrAttribute} & \textbf{Alias}  & \textbf{Constraints} &\textbf{\langStrNotes}\tabularnewline \hline\hline \endhead%
}

\newcommand{\DSPEtablebeginattributeConstraint}[7]%
{%
	\begin{small}%
	\begin{longtable}{|v{0.3\textwidth}|v{0.3\textwidth}|v{0.35\textwidth}|}%
    \caption{#2 \langStrAttributes}%
    \label{#3}%
    \hline%
    \textbf{\langStrAttribute} & \textbf{Constraint}  & \textbf{\langStrNotes}\tabularnewline \hline\hline%
    \endfirsthead%
    \hline%
    \textbf{\langStrAttribute} & \textbf{Constraint}  & \textbf{\langStrNotes}\tabularnewline \hline\hline \endhead%
}

\newcommand{\DSPEtablebeginmethod}[7]%
{%
	\ifthenelse{\equal{#4}{command interface}}%
	{%
		\begin{small}%
		\begin{longtable}{|v{0.3\textwidth}|v{0.7\textwidth}|}%
		\caption{#2 \langStrCommands}%
		\label{#3}%
		\hline%
		\textbf{\langStrCommand} & \textbf{\langStrDescription} \tabularnewline \hline\hline%
		\endfirsthead%
		\hline%
		\textbf{\langStrDescription} & \textbf{\langStrDescription} \tabularnewline \hline\hline \endhead%
	}{%
		\begin{small}%
		\begin{longtable}{|v{0.4\textwidth}|v{0.2\textwidth}|v{0.35\textwidth}|}%
		\caption{#2 \langStrMethods}%
		\label{#3}%
		\hline%
		\textbf{\langStrMethod} & \textbf{\langStrReturnType} & \textbf{\langStrNotes} \tabularnewline \hline\hline%
		\endfirsthead%
		\hline%
		\textbf{\langStrMethod} & \textbf{\langStrReturnType} & \textbf{\langStrNotes} \tabularnewline \hline\hline \endhead%
	}%
}

\newcommand{\DSPEtablebeginreqtrace}[7]%
{%
	\begin{small}%
	\begin{longtable}{|v{0.5\textwidth}|v{0.5\textwidth}|}%
	\caption{#2}%
	\label{#3}%
	\hline%
	\textbf{\langStrRequirement} & \textbf{\langStrRealizedBy} \tabularnewline \hline\hline%
	\endfirsthead%
	\hline%
	\textbf{\LangStrRequirement} & \textbf{\langStrRealizedBy} \tabularnewline \hline\hline \endhead%
}

\newcommand{\DSPEtablebeginrunstate}[7]%
{%
    \begin{small}%
    \begin{longtable}{|v{0.5\textwidth}|v{0.5\textwidth}|}%
    \caption{#2}%
    \label{#3}%
    \hline%
    \textbf{Variable} & \textbf{\langStrValue} \tabularnewline \hline\hline%
	\endfirsthead%
	\hline%
	\textbf{Variable} & \textbf{\langStrValue} \tabularnewline \hline\hline \endhead%
}

\newcommand{\DSPEtablebeginsearchtwo}[7]%
{%
	\begin{small}%
    \begin{longtable}{|v{0.5\textwidth}|v{0.5\textwidth}|}%
    \caption{#2}%
    \label{#3}%
    \hline%
    \textbf{#5} & \textbf{#6} \tabularnewline \hline\hline%
    \endfirsthead%
	\hline%
	\textbf{#5} & \textbf{#6} \tabularnewline \hline\hline \endhead%
}

\newcommand{\DSPEtablebeginsearchtwolandscape}[7]%
{%
    \begin{landscape}%
	\begin{small}%
    \begin{longtable}{|v{0.39\paperheight}|v{0.39\paperheight}|}%
    \caption{#2}%
    \label{#3}%
    \hline%
    \textbf{#5} & \textbf{#6} \tabularnewline \hline\hline%
    \endfirsthead%
	\hline%
	\textbf{#5} & \textbf{#6} \tabularnewline \hline\hline \endhead%
}

\newcommand{\DSPEtablebeginsearchthree}[7]%
{%
    \begin{small}%
    \begin{longtable}{|v{0.3\textwidth}|v{0.3\textwidth}|v{0.3\textwidth}|}%
    \caption{#2}%
    \label{#3}%
    \hline%
    \textbf{#5} & \textbf{#6} & \textbf{#7} \tabularnewline \hline\hline%
    \endfirsthead%
	\hline%
	\textbf{#5} & \textbf{#6} & \textbf{#7} \tabularnewline \hline\hline%
    \endhead%
}

\newcommand{\DSPEtablebeginsearchthreelandscape}[7]%
{%
	\begin{landscape}%
    \begin{small}%
    \begin{longtable}{|v{0.26\paperheight}|v{0.26\paperheight}|v{0.26\paperheight}|}%
    \caption{#2}%
    \label{#3}%
    \hline%
    \textbf{#5} & \textbf{#6} & \textbf{#7} \tabularnewline \hline\hline%
    \endfirsthead%
	\hline%
	\textbf{#5} & \textbf{#6} & \textbf{#7} \tabularnewline \hline\hline%
    \endhead%
}
% Begin der Table-Umgebung (sonderfall für tabellen in rtf dokumenten)
% Erhaelt als Argumente
% #1 Caption
% #2 Guid
% #3 Stereotype
% #4 Notes
\newcommand{\DSPEtablebeginexttab}[4]%
{%
    \begin{table}[h]%
	\caption{#1}%
    \label{#2}%
	\ifthenelse{\equal{#4}{ }}{\medskip}%
	{%
	\centering
	\DSPEnote{#4}%
	\bigskip%
	}
}

% Ende der Table-Umgebung
% Erhaelt als Argumente:
% #1 Caption
% #2 Guid
% #3 Stereotype
% #4 landscape
\newcommand{\DSPEtableend}[4]%
{%
	\end{longtable}%
	\end{small}%
	\ifthenelse{\equal{#4}{landscape}}%
	{%
	\end{landscape}%
	}{}%
}
\newcommand{\DSPEtableendexttab}%
{%
	\end{table}%
}
% Tabellen aus CSV Dateien
% Erhält als Argument:
% #1 Caption
% #2 Guid
% #3 Dateipfad
% #4 CSV Delimiter
% #5 Notes
\newcommand{\DSPEcsvtable}[5]%
{%
	\DTLsetseparator{#4}%
	\DTLloaddb{csvdata}{\detokenize{#3}}%
	\DTLdisplaylongdb[caption={#1}, contcaption={#1 (\langStrContinued)},label={#2}]{csvdata}%	
	\ifthenelse{\equal{#5}{ }}{\medskip}%
	{%
		\begin{center}%
		\DSPEnote{#5}%
		\bigskip%
		\end{center}%
	}%
}
% Referenz auf Tabellen
% Wird derzeit nicht genutzt, da die Tabellen derzeit nicht floating sind
\newcommand{\DSPEtabref}[3]%
{%
}

% Erhält als Argumente
% #1 Namen
% #2 Typ
% #3 Notes des Klassen-Attributs
% #4 Stereotype
% #5 Default-Wert
% #6 Attribut-Style
% #7 Attribut-Constraints
% #8 Guid des Attributs
% #9 multiplicity
\newcommand{\DSPEattribute}[9]%
{%
        \ifthenelse{\equal{#6}{}}%
        {%
		\ifthenelse{\equal{#7}{}}%
	        {%
			\textbf{#1 \label{#8}} #2 #9 & #3\ifthenelse{\equal{#5}{}}{}{~Default: #5}\tabularnewline \hline%
		}{%
			\textbf{#1 \label{#8}} #2 #9 & #7 & #3\ifthenelse{\equal{#5}{}}{}{~Default: #5}\tabularnewline \hline%
		}%
        }{%
		 \ifthenelse{\equal{#7}{}}%
                {%
			\textbf{#1 \label{#8}} #2 #9 & #6 & #3\ifthenelse{\equal{#5}{}}{}{~Default: #5}\tabularnewline \hline%
		}{%
			\textbf{#1 \label{#8}} #2 #9 & #6 & #7 & #3\ifthenelse{\equal{#5}{}}{}{~Default: #5}\tabularnewline \hline%
		}%
	}%
}

\newcommand{\DSPErunstate}[3]%
{%
	\textbf{#1} & #2 \tabularnewline \hline%
}

\newcommand{\DSPEsearchtwo}[2]%
{%
	#1 & #2 \tabularnewline \hline%
}
\newcommand{\DSPEsearchthree}[3]%
{%
	#1 & #2 & #3\tabularnewline \hline%
}

% Erhält als Argumente den namen und die Notes der Constraint.
\newcommand{\DSPEconstraint}[2]{#1: #2}
%Zeilenumbruch innerhalb einer Tabelle
\newcommand{\DSPEconstraintsep}%
{%
\newline%
}
% Erhält als Argumente
% #1 Namen der Methode
% #2 Rückgabewert der Methode
% #3 Methoden-Parameter
% #4 Notes
% #5 Guid der Methode
% #6 Stereotype des Klassenelements
\newcommand{\DSPEmethod}[6]%
{%	
	\ifthenelse{\equal{#6}{command interface}}%
	{%
	\textbf{#1 \label{#5}} & #4 \tabularnewline \hline%
	}{%
	\textbf{#1(#3) \label{#5}} & #2 & #4 \tabularnewline \hline%
	}
}

% Für Funktionsbeschreibungen , z.B. class::method()
\newcommand{\DSPEfuncdescr}[1]
{%
	\noindent\textbf{#1}\\
	\hspace{2mm}
}

% Requirement Trace
% Erhaelt als Argumente:
% #1 den Namen der Realisierten Anfoderderung
% #2 eine DSPEreqtrref
\newcommand{\DSPEreqtr}[2]%
{%
	#1 & #2  \tabularnewline \hline%
}
% Requirement Trace Referenz
% Erhaelt als Argumente:
% #1 Den Namen des realisierenden Elementes
% #2 die Guid des realisierenden Elementes
\newcommand{\DSPEreqtrref}[2]%
{%
	#1 (\langStrseeSection{}~\ref{#2}) %
}

